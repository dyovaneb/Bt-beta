\documentclass[oneside,12pt]{amsart}
\usepackage[utf8]{inputenc}
\usepackage{xcolor}
\usepackage{amsthm}
\usepackage{amsmath}
\usepackage{amssymb}



\definecolor{dblue}{rgb}{0.0, 0.0, 0.55}
\definecolor{burgundy}{rgb}{0.5, 0.0, 0.13}
\definecolor{dgreen}{rgb}{0.13, 0.55, 0.13}

\usepackage[colorlinks, urlcolor=black]{hyperref}
% \hypersetup{citecolor=dblue, linkcolor=dblue}

%\setlenght{\parindent}{3ex}
\setlength{\textwidth}{6.5in}
\setlength{\textheight}{9in}
\setlength{\topmargin}{0in}
\setlength{\oddsidemargin}{0in}
\setlength{\parindent}{0pt}

\newtheorem{theorem}{Teorema}[section]
\newtheorem{lemma}[theorem]{Lema}
\newtheorem{proposition}[theorem]{Proposition}
\theoremstyle{definition}
\newtheorem{defi}[theorem]{Definición}
\newtheorem{corollary}[theorem]{Corollary}
\newtheorem{remark}[theorem]{Observación}
\newtheorem{ejemplo}[theorem]{Ejemplo}

\numberwithin{equation}{section}

\begin{document}



 \section{Sistema de ra\'ices af\'ines } Sea $V$ un espacio $\mathbb{R}$-vectorial y $A$ es un espacio afín sobre $V$.  Sea $\Psi$ un sistema de ra\'iz af\'in de $A^*$.  Recordamos la definición de sistema de raíces afín.

 \begin{defi}
    Un subconjunto $\Psi$ de $A^*$ es un sistema de raíz afín, si
    \begin{enumerate}
        \item $\Psi$ genera $A^*$ y no contiene funciones constante.
        \item Para todo $\alpha\in\Psi$ hay un $\dot{\alpha}^\vee\in V\subset A^{**}$ tal que $\langle\alpha,\dot{\alpha}^\vee \rangle=2$ y la reflexi\'on $r_\alpha(x)=x-\langle x,\alpha \rangle\cdot \dot{\alpha}^\vee$ preserva $\Psi$.
        \item Para $\alpha,\beta\in \Psi$, $\langle\alpha,\dot{\beta}^\vee \rangle\in \mathbb Z$,
        \item El subgrupo $W(\Psi)$ de ${\rm Aut}(A)$ dado por $\{r_\alpha\colon \alpha\in \Psi\}$ actua propiamente, i.e. si $K_1,K_2\subset A$ son compactos, entonces $\{w\in W(\Psi)\colon wK_1\cap K_2\neq\empty\}$ es finito.
    \end{enumerate}
    Diremos que $\Psi$ es reducido, si $\alpha,\beta\in \Psi$ y $\alpha=r\cdot\beta$ con $r\in \mathbb Z$, entonces $r\in\{-1,1\}$.
\end{defi}


\section{Ejemplos}
Sea $\Phi$ un sistema de raíz de $V^*$. El sistema de raiz af\'in asociado $\Psi_\Phi$ est\'a definido por
    \[\{(a,k)\colon a\in \Phi, k\in I_\alpha\},\]
    donde $I_a$ es $\mathbb Z$ si $a$ no es divisible o es $2\mathbb Z+1$, si no.
    Entonces $\Psi_\Phi$ es un sistema de raíz afín reducido en $V^*\oplus k$.

    \subsection{Arreglo de hiperplanos: habitaciones y facetas} Sea $\psi\in \Psi$ una ra\'iz af\'in.
    Un \textit{hiperplano de ra\'iz afín asociado a $\psi$} de $A$ es el hiperplano af\'in dado por $H_\psi = \{x \in A: \psi(x) = 0\}$, donde $\psi \in \Psi$. Consideramos el arreglo de hiperplanos dado por $\{H_\psi\colon \psi\in \Psi\}$, lo que da una partici\'on en el espacio af\'in $A$.
    
    
    Las partes que no intersectan ningun hiperplano se llama \emph{habitaciones} que no intersecatan ningun hiperplano. En general las partes se llaman \emph{facetas}.


\section{Habitaciones y raices afines simples}

Dada una habitación $\mathcal{C}$. Denotamos $\psi \in \Psi$ se llama \textit{positiva con respecto a $\mathcal C$} (respectivamente \textit{negativa con respecto a $\mathcal C$}) si $\psi(x) > 0$ (respectivamente $\psi(x) < 0$) para todo $x \in \mathcal{C}$. Denotamos por $\Psi(\mathcal{C})^+$ y $\Psi(\mathcal{C})^-$ el conjunto de raíces afines positivas y negativas, respectivamente. Observamos que tenemos la unión disjunta $\Psi = \Psi(\mathcal{C})^+ \cup \Psi(\mathcal{C})^-$.
   
 Sea $\mathcal{C}$ una cámara. Sea $\Psi(\mathcal{C})^0$ el conjunto que consiste en aquellas $\psi \in \Psi(\mathcal{C})^+$ indivisibles para las cuales $H_\psi$ es una pared de $\mathcal{C}$. El conjunto de estas raices afines se llama una \textit{base} de $\Psi$, y sus elementos se llaman \textit{raíces afines simples}.
     
 Observe que $\mathcal{C}$ está determinada de manera única por $\Psi(\mathcal{C})^0$, concretamente como la intersección de los semiespacios $A^{\psi>0}$ para $\psi \in \Psi(\mathcal{C})^0$.


\section{Proposici\'on}
Sea $\Phi$ un sistema de raíces reducido irreducible y $\Psi = \Psi_\Phi=\{a_1, \ldots, a_\ell\}$ el sistema de raíces afín asociado.  Si $a_1, \ldots, a_\ell$ es una base para $\Phi$, $a_0$ es la raíz más larga, y definimos $\psi_1 = 
    a_1, \ldots, \psi_\ell = a_\ell$ y $\psi_0 = 1$, entonces $\psi_v, \ldots, \psi_\ell$ es una base para $\Psi$. 







\subsection{Puntos especiales}



\section{Edificios de Bruhat-Tits: general} Sea $F$ un cuerpo henseliano de valuacion discreta, $k$ el cuerpo residual (perfecto) y $K=F^u$. Adem\'as sea $G$ un grupo reductivo sobre $F$.


Hay una acci\'on de $\Gamma={\rm Gal}(K/F)$ en $B(G/K)$, y todas las orbitas son finitas.
\[B(G/F)=B(G/K)^{\Gamma}.\]
Es no vacio, cerrado y convexo.
\begin{defi}Definimos
    \begin{enumerate}
        \item una $\Gamma$-faceta de $B$ es una faceta $\Gamma$ invariante.
        \item Una $\Gamma$-alcoba es un $\Gamma$-faceta maximal.
        \item Una $\Gamma$-v\'ertice es una $\Gamma$-faceta minimal.
\end{enumerate}
\end{defi}

\begin{defi}
    Una faceta de $B(G/F)$ es $\Omega^\Gamma$, para una $\Gamma$-faceta $\Omega$ de $B(G/K)$. Es una alcoba o v\'ertice si $\Omega$ es  una $\Gamma$-alcoba o ve\'rtices, respectivamente.
\end{defi}


Hasta ahora no queda claro si de esta manera tenemos propiedades similar como en el caso de cuasi escindido.

\subsubsection*{Grupos esquem\'aticos}
\subsubsection*{Subgrupos parahoricos}


\section{Aplicaciones}


\section*{Referencias}
\begin{thebibliography}{99}
    \bibitem{BourLie} N. Bourbaki, \emph{Groupes et algèbres de Lie
    Chapitres 4, 5 et 6}, Springer 2007
    \bibitem{KaGoBuild}T. Kaletha, G. Prasad, \emph{Bruhat–Tits theory: a new approach}
    New Mathematical Monographs, 44. Cambridge University Press.
    \bibitem{MilneGrp} J.S. Milne. \emph{Algebraic Groups: The Theory of Group Schemes of Finite Type over a Field}
    \bibitem{BrCnotes}B. Conrad. \emph{Algebraic groups I and II},
    disponible en \href{https://www.ams.org/open-math-notes/omn-view-listing?listingId=110662}{https://www.ams.org/open-math-notes/omn-view-listing?listingId=110662} y 
    \href{https://www.ams.org/open-math-notes/omn-view-listing?listingId=110663}{https://www.ams.org/open-math-notes/omn-view-listing?listingId=110663}
\end{thebibliography}
    


\end{document}

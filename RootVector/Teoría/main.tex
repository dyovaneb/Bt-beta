\documentclass[oneside,12pt]{amsart}
\usepackage[utf8]{inputenc}
\usepackage{xcolor}
\usepackage{amsthm}
\usepackage{amsmath}
\usepackage{amssymb}



\definecolor{dblue}{rgb}{0.0, 0.0, 0.55}
\definecolor{burgundy}{rgb}{0.5, 0.0, 0.13}
\definecolor{dgreen}{rgb}{0.13, 0.55, 0.13}

\usepackage[colorlinks, urlcolor=black]{hyperref}
% \hypersetup{citecolor=dblue, linkcolor=dblue}

%\setlenght{\parindent}{3ex}
\setlength{\textwidth}{6.5in}
\setlength{\textheight}{9in}
\setlength{\topmargin}{0in}
\setlength{\oddsidemargin}{0in}
\setlength{\parindent}{0pt}

\newtheorem{theorem}{Teorema}[section]
\newtheorem{lemma}[theorem]{Lema}
\newtheorem{proposition}[theorem]{Proposition}
\theoremstyle{definition}
\newtheorem{defi}[theorem]{Definición}
\newtheorem{corollary}[theorem]{Corollary}
\newtheorem{remark}[theorem]{Observación}
\newtheorem{ejemplo}[theorem]{Ejemplo}

\numberwithin{equation}{section}

\begin{document}



 \section{Sistema de ra\'ices af\'ines}
 

\begin{remark}
    Sea $\Phi$ un sistema de raíces reducido irreducible y $\Psi = \Psi_\Phi=\{a_1, \ldots, a_\ell\}$ el sistema de raíces afín asociado. 
     Si $a_1, \ldots, a_\ell$ es una base para $\Phi$, $a_0$ es la raíz más larga, y definimos $\psi_1 = 
    a_1, \ldots, \psi_\ell = a_\ell$ y $\psi_0 = 1$, entonces $\psi_v, \ldots, \psi_\ell$ es una base para $\Psi$. 
\end{remark}



\subsection{Puntos especiales}



\section{Edificios de Bruhat-Tits: general} Sea $F$ un cuerpo henseliano de valuacion discreta, $k$ el cuerpo residual (perfecto) y $K=F^u$. Adem\'as sea $G$ un grupo reductivo sobre $F$.


Hay una acci\'on de $\Gamma={\rm Gal}(K/F)$ en $B(G/K)$, y todas las orbitas son finitas.
\[B(G/F)=B(G/K)^{\Gamma}.\]
Es no vacio, cerrado y convexo.
\begin{defi}Definimos
    \begin{enumerate}
        \item una $\Gamma$-faceta de $B$ es una faceta $\Gamma$ invariante.
        \item Una $\Gamma$-alcoba es un $\Gamma$-faceta maximal.
        \item Una $\Gamma$-v\'ertice es una $\Gamma$-faceta minimal.
\end{enumerate}
\end{defi}

\begin{defi}
    Una faceta de $B(G/F)$ es $\Omega^\Gamma$, para una $\Gamma$-faceta $\Omega$ de $B(G/K)$. Es una alcoba o v\'ertice si $\Omega$ es  una $\Gamma$-alcoba o ve\'rtices, respectivamente.
\end{defi}


Hasta ahora no queda claro si de esta manera tenemos propiedades similar como en el caso de cuasi escindido.

\subsubsection*{Grupos esquem\'aticos}
\subsubsection*{Subgrupos parahoricos}


\section{Aplicaciones}


\section*{Referencias}
\begin{thebibliography}{99}
    \bibitem{BourLie} N. Bourbaki, \emph{Groupes et algèbres de Lie
    Chapitres 4, 5 et 6}, Springer 2007
    \bibitem{KaGoBuild}T. Kaletha, G. Prasad, \emph{Bruhat–Tits theory: a new approach}
    New Mathematical Monographs, 44. Cambridge University Press.
    \bibitem{MilneGrp} J.S. Milne. \emph{Algebraic Groups: The Theory of Group Schemes of Finite Type over a Field}
    \bibitem{BrCnotes}B. Conrad. \emph{Algebraic groups I and II},
    disponible en \href{https://www.ams.org/open-math-notes/omn-view-listing?listingId=110662}{https://www.ams.org/open-math-notes/omn-view-listing?listingId=110662} y 
    \href{https://www.ams.org/open-math-notes/omn-view-listing?listingId=110663}{https://www.ams.org/open-math-notes/omn-view-listing?listingId=110663}
\end{thebibliography}
    


\end{document}
